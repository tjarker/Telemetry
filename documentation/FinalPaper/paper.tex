\documentclass[conference]{IEEEtran}
\IEEEoverridecommandlockouts

\usepackage{cite}
\usepackage[pdftex]{graphicx}
\usepackage{cite}
\usepackage{amsmath,amssymb,amsfonts}
\usepackage{algorithmic}
\usepackage{graphicx}
\usepackage{textcomp}
\usepackage{xcolor}
\def\BibTeX{{\rm B\kern-.05em{\sc i\kern-.025em b}\kern-.08em
    T\kern-.1667em\lower.7ex\hbox{E}\kern-.125emX}}

\usepackage{url}

% correct bad hyphenation here
\hyphenation{op-tical net-works semi-conduc-tor}


\begin{document}

\title{Telemetry Module for Solar Vehicle}

\author{\IEEEauthorblockN{Victor Alexander Hansen s194027, Steffan Martin Kunoy s194006, Tjark Petersen, s186083}
\IEEEauthorblockA{\textit{Department of Electrical Engineering} \\
\textit{Technical Uniservity of Denmark}\\
Kongens Lyngby, Denmark \\
31015 Introductory project - electrotechnology\\
s194027@student.dtu.dk, s194006@student.dtu.dk, s186083@student.dtu.dk}}
%\author{Victor Alexander Hansen, Steffan Martin Kunoy, Tjark Petersen}

\maketitle

% As a general rule, do not put math, special symbols or citations
% in the abstract or keywords.
\begin{abstract}
Abstract
\end{abstract}

% Note that keywords are not normally used for peerreview papers.
\begin{IEEEkeywords}
ROAST, telemetry, CAN
\end{IEEEkeywords}



\section{Introduction}

\IEEEPARstart{I}{n} 2023, DTU Roadrunners Solar Team, abbreviated ROAST, is set to participate in the Bridgestone World Solar Challenge, a race spanning over 3000 km across Australia from Darwin in Northern Territory to Adelaide in South Australia. The solar car have limited opportunities for recharging during the race, so the main source of energy during the race, is from solar energy. To this end, solar panels will cover the car, however, the solar car is only allowed to have a maximum of 4 square meters of solar panels. This emphasizes the need to create an energy-efficient solar car to win the race \cite{wsc}.\\
Throughout the race the solar car will be monitored by a support car, which can monitor the car's condition and issue commands. This is a crucial function as the solar car will be navigating Australia's busy highways, where the heat and tire pressure may cause a detriment to the vehicle. The support car will therefore be able to analyze and react to the data stream being transmitted from the solar car, even when the driver is preoccupied by driving the car and navigating traffic. \\
The main function of the telemetry module is to read data from the solar car's on-board CAN (Controller Area Network) bus and transmit it via an RF-transceiver to the support car, while simultaneously logging data locally in a "black box" memory chip. The support car is equipped with a similar module such that the support crew can react to the data manually or automatically by sending commands back to the CAN bus. As a consequence, the telemetry module plays a vital role in the communication between both vehicles, considering the distance between them may be up to 1 km depending on traffic conditions.\\
Therefore, the telemetry module must both serve the purpose of fulfilling the necessary specifications for communication, while also being as energy-efficient as possible. The project will mainly focus on the former that is, ensuring that the module meets the specifications with a solid solution.

\subsection{Problem Statement}
Remote sensing of data from the solar car will give the ROAST team a competitive advantage as the support vehicle will take on a more active role in controlling and optimising the car's performance. This will ease the burden on the driver and engage a large part of the supporting crew. For these reasons we have chosen to focus our project on the following points:   
\begin{itemize}
    \item How can we design a module that can read and write sensor data and commands from a CAN bus network? 
    \item How can CAN data be transmitted securely over a distance of up to one kilometer?
    \item How does the support-car module process data upon receiving it from the solar car?
    \item What is the best method for storing data locally (black box) while being accessible at a later time?
    \item How can the CAN data be made accessible for further processing in e.g. Matlab?
\end{itemize}


\section{Methodology}
Sprint method and verification of software
\section{Design}
\begin{itemize}
    \item Block diagram
    \item Components
    \item Software stack
    \item Final design
\end{itemize}
\subsection{CAN} % Victor
\subsection{nRF24 \& Shockburst} % Steffan 
\subsection{Message protocol}

\section{Implementation}

\subsection{Solar Car} 
\subsubsection{Black Box} % tjark
\subsubsection{CAN transceiver}
\subsection{Support Vehicle}
\subsubsection{GUI} % tjark
\subsubsection{RF transceiver}

\section{Verification} % Victor
Throughout the project we have used the cross-platform IDE PlatformIO, which is a development platform for embedded systems. This has greatly increased the efficiency in our work due to the 

\section{Discussion}

\section{Conclusion}
The conclusion goes here.


% use section* for acknowledgment
\section*{Acknowledgment}


The authors would like to thank Christian Kampp Kruse for his assistance as coordinator for this project. The authors would also like to thank Claus from the department of Mechanical Engineering at DTU. Finally, we 


\bibliographystyle{IEEEtran}
\bibliography{IEEEabrv,../bib/paper}

\section{References}
\begingroup
\renewcommand{\section}[2]{}%
%\renewcommand{\chapter}[2]{}% for other classes
\begin{thebibliography}{}
\bibitem{wsc}
Bridgestone World Solar Challenge, 
URL: https://worldsolarchallenge.org/
\bibitem{ROAST}
Reports and source files provided by DTU ROAST
\bibitem{teensy}
Teensy microcontroller,
URL: https://www.pjrc.com/teensy/
\bibitem{C++}
Website for learning C++
https://www.learn-cpp.org/
\end{thebibliography}
\endgroup


\end{document}
