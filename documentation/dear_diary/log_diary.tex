\documentclass[]{article}
\usepackage[utf8]{inputenc}
\usepackage[danish,english]{babel}
\usepackage{graphicx}
\usepackage[usenames]{xcolor}
\usepackage{times}
\usepackage{listings}
\usepackage{textcomp}
\usepackage{hyperref}
\usepackage{textpos}
\usepackage{pgfgantt}
\usepackage{lscape}
\usepackage{indentfirst}
\usepackage{tikz}
\usepackage{pgfplots}
\usepackage{ifthen}
\usepackage{float}
\usepackage{pict2e}
\usepackage[export]{adjustbox}
\pgfplotsset{compat=1.13}
\usepackage{siunitx}
\usepackage{tabularx}

% adjustment to page format
\marginparwidth=0pt
\oddsidemargin=0pt
%\evensidemargin=0pt
\marginparsep=0pt
%\topmargin=-15 mm
\textwidth=168 mm
\textheight=210 mm
\parskip 0.27 em
\parindent=0pt
%\leftmargin=1cm
\pagenumbering{arabic}
% Define DTU color
\definecolor{dtured}{RGB}{153,0,0}
\title{Projektplan}

\begin{document}
This log shall be used as a technical diary we write throughout the course. It will be used to write the final report more easily.

\section{05-03}
We've set up PlatformIO in Visual Studio Code as the environment in which we write our code.

\section{10-03}
We've made an adjustment in the ressources and will be using a teensyboard 3.6 instead of a 4.1, since the 3.6 board is already in stock at DTU. It has a slower clock (180Mhz, compared to 4.1's 600Mhz), but we judge that this won't matter to the project solution.

\section{12-03}
We've decided to divide the work among us.
Tjark will be having the 3.6 Teensy board with the SD module, Steffan will be having the 4.0 Teensy board and Victor will work on the related software.
As Tjark has a soldering station, he will take care of soldering the components, if this is not done in laboratory.

More specifically, Steffan will look into RF, Tjark into SD storage and Victor at CANbus conncection.

\end{document}